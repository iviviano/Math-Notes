\documentclass[12pt, reqno]{amsart}

%\usepackage{upgreek}
\usepackage[margin=3.5 cm]{geometry}
\usepackage{graphicx, mathabx}
\usepackage{color}
%\usepackage{subfigure}
\newtheorem{theorem}{Theorem}[section]
\newtheorem*{theorem*}{Theorem}          %theorem without number
\newtheorem{prop}{Proposition}[section]
\newtheorem*{prop*}{Proposition}  % proposition without number
\newtheorem{coro}{Corollary}[section]
\newtheorem{lemma}[theorem]{Lemma}
\newtheorem{conj}{Conjecture}[section]
\newtheorem{obs}{Observation}[section]

\theoremstyle{definition}
\newtheorem{definition}[theorem]{Definition}
\newtheorem{example}[theorem]{Example}
\newtheorem{xca}[theorem]{Exercise}

\theoremstyle{remark}
\newtheorem{remark}[theorem]{Remark}

%\numberwithin{equation}

%    Absolute value notation
\newcommand{\abs}[1]{\lvert#1\rvert}
\newcommand{\norm}[1]{\lVert#1\rVert}
\DeclareMathOperator{\re}{Re}
\DeclareMathOperator{\im}{Im}
\newcommand{\ud}{\mathrm{d}}

\begin{document}

\title[Math 357 - Harmonic Analysis]{Problem set no. 2 - Isaac Viviano}


\maketitle

\newpage
\null
\newpage

\section{Problems:} 

\begin{itemize}

\item {\bf{Problem 1 - The vibrating string \& harmonics:}} 

\vspace{0.1 cm}
\begin{itemize}
\item[(b)]
    Suppose $y(x,t)=f(x)\cdot g(t)$ is a solution to the 1-dimensional wave equation: \begin{equation}\frac{\partial ^{2}y}{\partial t^{2}}=c^{2} \frac{\partial ^{2}y}{\partial x^{2}}\end{equation}Note that \begin{align*}
        \frac{\partial y}{\partial t}&= \frac{\partial }{\partial t}(f(x)\cdot g(t))=f(x)\cdot\frac{\partial }{\partial t}g(t)=f(x)g'((t)\\
        \frac{\partial ^{2}y}{\partial t^{2}}&= f(x)g''(t)\\
        \frac{\partial ^{2}y}{\partial x^{2}}&= f''(x)g(t)
        \end{align*}
        So, $$f(x)g''(t)=c^{2}f''(x)g(t)$$On $f(x)\ne0$ and $g(t)\ne 0$, this may be written \begin{equation}\frac{g''(t)}{g(t)}=c^{2}\frac{f''(x)}{f(x)}\end{equation}Note that if $f(x)=0$ or $g(t)=0$, $y(x,t)=0$, so $y$ satisfies (1). So, any separated solution $y$ to (1) must only satisfy (2).
        
        Suppose $y(x,t)=f(x)g(t)$ is a solution to (1). Fix $t_{0}\in \mathbb{R}$ with $g(t_{0})\ne0$ and let $$-k= \frac{g''(t_{0})}{g(t_{0})}$$Then, for all $x\in \mathbb{R}$ with $f(x)\ne0$, \begin{equation}-k=c^{2} \frac{f''(x)}{f(x)}\end{equation}So, $f$ is a solution to (2). Similarly, fixing an $x$ shows that $g$ is a solution to \begin{equation}-k= \frac{g''(t)}{g(t)}\end{equation}
\vspace{0.1 cm}

\vspace{0.1 cm}
\item[(c)] We now apply the boundary condition $y(0,t)=y(l,t)=0$. We have $y(x,t)=f(x)\cdot g(t)$. If $f(0)=0$ and $f(l)=0$, then for all $t$, \begin{align*}
    y(0,t)&= f(0)\cdot g(t)=0\cdot g(t)=0\\
    y(l,t)&= f(l)\cdot g(t)=0\cdot g(t)=0
    \end{align*}
    so the boundary conditions are satisfied. 
    
    Suppose $f(0)\ne0$. We need that for all $t$, $y(0,t)=0$: $$0=y(0,t)=f(0)\cdot g(t)\implies g(t)=0$$But this is only possible if $g$ is identically 0. This gives $f$ identically 0, not a solution we are interested in. The same argument shows that $f(l)=0$ for any separated solution to the initial value problem.
    
    Thus, $y(x,t)=f(x)\cdot g(t)$ is a solution to the initial value problem (1) if and only if $f(0)=f(l)=0$. 
    
    

\vspace{0.1 cm}
\item[(d)] Solve: $$f''+kf=0$$
From Set 1, the general solution is $$f(x)=A\cos \sqrt{k}x+B\sin \sqrt{k}x$$
Suppose there is a nonzero solution $f$ that satisfies the initial value problem $f(0)=f(l)=0$. Then, \begin{align*}
0=f(0)&= A\cos \sqrt{k}0+B\sin \sqrt{k}0\\
&= A
\end{align*}
So, $A=0$. The second boundary condition gives \begin{align*}
0=f(l)&= B\sin(\sqrt{k}l)\\
\sin(\sqrt{k}l)&= 0
\end{align*}
This occurs when $\sqrt{k}\cdot l=\pi\cdot n$ for some integer $n$. Solving for $k$, we see that possible values are $$k= \left(\frac{n\pi}{l}\right)^{2}$$

Let $k_{n}=\left(\frac{n\pi}{l}\right)^{2}$. Then, for all $n$, the function $f_n$ defined by $$f_{n}(x)=\sin \sqrt{k}x$$is a solution to the initial value problem: \begin{align*}
f_{n}'(x)&= \sqrt{k}\cos \sqrt{k}x\\
f_{n}''(x)&= -k\sin \sqrt{k}x\\
f_{n}(0)&= 0\\
f_{n}(l)&= \sin \sqrt{k}l=\sin\left(\frac{n\pi}{l}\cdot l\right)=0\\
f''(x)+kf(x)&= -k\sin \sqrt{k}x+k\sin \sqrt{k}x=0
\end{align*}

\vspace{0.1 cm}
\item[(e)] 
Solving the time differential equation, $$g''+kc^{2}g=0$$we get the general solution $$g(x)=A\cos c\sqrt{k}t+B\sin c\sqrt{k}t$$which may be rewritten $$g(x)=B\sin(c\sqrt{k}t+\phi)$$

The general solution to the wave equation is the product of the these two solutions for the same $k$ value. So, $$y=C\sin\left(\frac{cn\pi t}{l}+\phi\right)\sin\frac{n\pi t}{l}$$where $C\in \mathbb{R}$ is a constant that represents the product of the temporal and spacial amplitudes. 


\vspace{0.2 cm}
\item {\bf{Problem 2 - Bernoulli solutions \& dividing the string:}} 

%Here is the code for including a picture. You first need to create your picture and save it as a pdf file. Supposing that this file is called ``fig_harmonics.pdf'' this would be the code that includes the picture. You have to remove that % at the beginning of the line. Writing % treats what is written as a comment; once you delete the % sign, it treats the text as part of your TeX code.

%\begin{figure}[htbp]
%\includegraphics[width= 0.8 \textwidth]{./fig_harmonics.pdf}
%\end{figure}

% NOTE: By changing the number in front of \textwidth you can scale the size of the figure (0.8 \textwidth means that the figure will use your picture to fill 0.8 times the width of the text).


This will restrict to the Bernoulli solutions with a node at $x= \frac{l}{N}$. We need Bernoulli solutions with $$f_{n}\left(\frac{l}{N}\right)=0$$Solving $$0=\sin\left(\frac{n \pi x}{\frac{l}{N}}\right)=\sin\left(\frac{nN\pi x}{l}\right)$$we get $f_k$ where $k=nN$ for some $n\in \mathbb{N}$. These are the Bernoulli solutions where $k$ is divisible by $N$.



\vspace{0.2 cm}
\item {\bf{Problem 4 - Ces\`aro means:}} 

Let $a_{n}$ be a sequence in $\mathbb{C}$ with $a_{n}\rightarrow a\in \mathbb{C}$. Let $$C_{n}= \frac{1}{n}\sum_{k=1}^{n}a_{k}$$
Let $\epsilon>0$ be given and pick $M\in \mathbb{N}$ such that for all $n\ge M$, $$|a_{n}-a|< \frac{\epsilon}{2}$$
Choose $N\ge M$ such that $$\frac{1}{N}\sum_{k=1}^{M}|a_{k}-a|< \frac{\epsilon}{2}$$
Then, if $n\ge N$,
\begin{align*}|C_{n}-a|&= \left| \frac{1}{n}\sum_{k=1}^{n}a_{k}-a\right|\\
&= \left| \frac{1}{n}\sum_{k=1}^{n}a_{k}- \frac{1}{n}\sum_{k=1}^{n} a\right|\\
&= \left| \frac{1}{n}\sum_{k=1}^{n}(a_{k}-a)\right|\\
&\le \frac{1}{n}\sum_{k=1}^{n}\left|a_{k}-a\right|\quad (\Delta\text{ inequality in }\mathbb{C})\\
&= \frac{1}{n}\sum_{k=1}^{M}|a_{k}-a|+ \frac{1}{n}\sum_{k=M+1}^{n}\left|a_{k}-a\right|\\
&\le \frac{1}{N}\sum_{k=1}^{M}\left|a_{k}-a\right|+ \frac{1}{n}\sum_{k=M+1}^{n}\left|a_{k}-a\right|\\
&< \frac{\epsilon}{2}+ \frac{(n-N)}{n}\cdot \frac{\epsilon}{2}\\
&\le \frac{\epsilon}{2}+\frac{\epsilon}{2}\\
&= \epsilon
\end{align*}
So, $C_n\to a$.

\vspace{0.2 cm}
\item {\bf{Problem 5 - Fourier coefficients \& derivatives:}} 



\vspace{0.1 cm}
\begin{itemize}
\item[(a)] 
Note that $$
\left|e^{-2\pi int}\right|= 1
$$for all $t\in \mathbb{R}$, and for a 1-periodic function $f$, $$\|f\|_{\infty}=\sup\{f(x):x\in[0,1]\}$$
Then, 
\begin{align*}
|\hat{f_{n}}|&= \left|\int_{0}^{1}e^{-2\pi int}f(t)\ dt\right|\\
&\le \int_{0}^{1}\left| e^{-2\pi int}f(t)\right|dt\quad(\Delta \text{ inequality})\\
&= \int_{0}^{1}\left|e^{-2\pi int}\right|\cdot \left|f(t)\right|\ dt\\
&= \int_{0}^{1}|f(t)|\ dt\\
&\le 1\cdot\sup_{x\in[0,1]}f(x)\quad(\text{ML estimate})\\
	&= \|f\|_{\infty}
\\
\end{align*}

\vspace{0.1 cm}
\item[(b)] 
Integration by parts: $$\int u(t)v'(t)\ dt= u(t)v(t)-\int u'(t)v(t)\ dt\quad(\text{IBP})$$
Let $u(t)=f(t)$ and $v'(t)=e^{-2\pi int}$. By problem (3), $$v(t)=\frac{e^{-2\pi int}}{-2\pi in}$$is an antiderivative of $v$. Note that \begin{align*}
v(0)&= \frac{-1}{2\pi in}\\
v(1)&= -\frac{e^{-2\pi in}}{{2\pi in}}
\end{align*}

\vspace{0.1 cm}
\item[(c)] 
let $P(m)$ be that $$\hat{f_{n}}= \left(\frac{1}{2\pi in}\right)^{m}\left(\hat{f_{n}}^{(m)}+\sum_{l=0}^{m-1}(2\pi in)^{m-1-l}(f^{(l)}(0)-e^{-2\pi in}f^{(l)}(1))\right)$$


\textbf{Base Case:} $m=1$

\begin{align*}
\frac{1}{2\pi in}\left(\hat{f'_{n}}+\sum_{l=0}^{0}(2\pi in)^{0-l}(f^{(l)}(0)-e^{-2\pi in}f^{(l)}(1))\right)&= \frac{1}{2\pi in}(\hat{f_{n}'}+f(0)-e^{-2\pi in}f(1))
\end{align*}
which equals $\hat{f_{n}}$ by (b), so $P(1)$ is true.

\textbf{Inductive Step:} Assume $P(m)$ for some $1\le m<k$

Then, 
\tiny
\begin{align*}
\hat{f_{n}}^{(m)}&= \frac{f^{(m)}(0)-e^{-2\pi in}f^{(m)}(1)+\hat{f_{n}}^{(m+1)}}{2\pi in}\quad (\text{part b})\\
\hat{f_{n}}&= \left(\frac{1}{2\pi in}\right)^{m}\left(\hat{f_{n}}^{(m)}+\sum_{l=0}^{m-1}(2\pi in)^{m-1-l}(f^{(l)}(0)-e^{-2\pi in}f^{(l)}(1))\right)\quad(\text{ind. hyp.})\\
&= \left(\frac{1}{2\pi in}\right)^{m}\left(\frac{f^{(m)}(0)-e^{-2\pi in}f^{(m)}(1)+\hat{f_{n}}^{(m+1)}}{2\pi in}+\sum_{l=0}^{m-1}(2\pi in)^{m-1-l}(f^{(l)}(0)-e^{-2\pi in}f^{(l)}(1))\right)\\
&= \left(\frac{1}{2\pi in}\right)^{m+1}\left(\hat{f_{n}}^{(m+1)}+f^{(m)}(0)-e^{-2\pi in}f^{(m)}(1)+\sum_{l=0}^{m-1}(2\pi in)^{m-l}(f^{(l)}(0)-e^{-2\pi in}f^{(l)}(1))\right)\\
&= \left(\frac{1}{2\pi in}\right)^{m+1}\left(\hat{f_{n}}^{(m+1)}+(2\pi i n)^{m-m}(f^{(m)}(0)-e^{-2\pi in}f^{(m)}(1))+\sum_{l=0}^{m-1}(2\pi in)^{m-l}(f^{(l)}(0)-e^{-2\pi in}f^{(l)}(1))\right)\\
&= \left(\frac{1}{2\pi in}\right)^{m+1}\left(\hat{f_{n}}^{(m+1)}+\sum_{l=0}^{m}(2\pi in)^{m-l}(f^{(l)}(0)-e^{2\pi in}f^{(l)}(1))\right)\\
&= \left(\frac{1}{2\pi in}\right)^{m+1}\left(\hat{f_{n}}^{(m+1)}+\sum_{l=0}^{(m+1)-1}(2\pi in)^{(m+1)-1-l}(f^{(l)}(0)-e^{2\pi in}f^{(l)}(1))\right)
\end{align*}
\normalsize
So, $P(m+1)$.

Therefore, by (PMI), $P(m)$ for all $1\le m\le k$. In particular, $$\hat{f_{n}}= \left(\frac{1}{2\pi in}\right)^{k}\left(\hat{f_{n}}^{(k)}+\sum_{l=0}^{k-1}(2\pi in)^{k-1-l}(f^{(l)}(0)-e^{-2\pi in}f^{(l)}(1))\right)$$
Now, \begin{align*}
&\left|\hat{f_{n}}^{(k)}+\sum_{l=0}^{k-1}(2\pi in)^{k-1-l}(f^{(l)}(0)-e^{-2\pi in}f^{(l)}(1))\right|\\
&\le \left|\hat{f_{n}}^{(k)}\right|+\left|\sum_{l=0}^{k-1}(2\pi in)^{k-1-l}(f^{(l)}(0)-e^{-2\pi in}f^{(l)}(1))\right|\\
&\le \left\|f^{(k)}\right\|+\left|\sum_{l=0}^{k-1}(2\pi in)^{k-1-l}(f^{(l)}(0)-e^{-2\pi in}f^{(l)}(1))\right|\\
:&= C\in \mathbb{R}
\end{align*}
Then, \begin{align*}
\left|\hat{f_{n}}\right|&\le \left| \left(\frac{1}{2\pi in}\right)^{k}\cdot C\right|= \frac{C}{|n|^{k}}
\end{align*}

\end{itemize}

\newpage
\vspace{0.2 cm}
\item {\bf{Problem 6 - Real Fourier series and the sawtooth function:}}

\vspace{0.1 cm}
\begin{itemize}


\vspace{0.1 cm}
\item[(b)] 
Let $\phi$ be the $2\pi$-periodic function described by $$\phi(\theta)= \frac{\pi-\theta}{2}$$and $\phi(0)=\phi(2\pi)=0$. 
For all $n\ne0,$
\begin{align*}
\int_{0}^{2\pi}e^{-ni\theta}d\theta&= - \left.\frac{e^{-ni\theta}}{-ni}\right|_{0}^{2\pi}=0\\
\int_{0}^{2\pi} \frac{\theta}{2}e^{-ni \theta}\ d\theta&= \left.\frac{-\theta e^{-ni\theta}}{2ni}\right|^{2\pi}_{0}-\int_{0}^{2\pi} \frac{1}{2}e^{-ni\theta}\ d \theta\\
&= \frac{-2\pi}{2ni}=- \frac{\pi}{ni}\\
\hat{f_{n}}&= \int_{0}^{2\pi}\phi(\theta)e^{-ni \theta}\ d\theta\\
&= \int_{0}^{2\pi} \frac{\pi-\theta}{2}e^{-ni \theta}\ d \theta\\
&= \int_{0}^{2\pi} \frac{\pi}{2}e^{-ni \theta}d \theta-\int_{0}^{2\pi} \frac{\theta}{2}e^{-ni \theta}\ d \theta\\
&= \frac{\pi}{ni}
\end{align*}
And for $n=0$, \begin{align*}
\hat{f_{0}}&= \int_{0}^{2\pi}\phi(\theta)e^{-ni\theta}\ d\theta=\int_{0}^{2\pi} \frac{\pi-\theta}{2}d\theta\\
&=\left. \frac{\pi}{2}\theta- \frac{\theta^{2}}{4}\right|_{0}^{2\pi}\\
&= \pi^{2}- \frac{(2\pi)^{2}}{4}=0
\end{align*}

So, the Fourier series is \begin{align*}
f(t)&= \sum_{n=-1}^{-\infty}\hat{f_{n}}e^{int}+\hat{f_{0}}+\sum_{n=1}^{\infty}\hat{f_{n}}e^{int}\\
&= \sum_{n=-1}^{-\infty}\hat{f_{n}}e^{int}+0+\sum_{n=1}^{\infty}\hat{f_{n}}e^{int}\\
&=\sum_{n=1}^{\infty}\hat{f_{-n}}e^{-int}+\sum_{n=1}^\infty\hat{f_{n}}e^{int}\\
&=\sum_{n=1}^{\infty} \frac{\pi}{-ni}(\cos (-nt)+i\sin(-nt))+\sum_{n=1}^{\infty} \frac{\pi}{ni}(\cos nt+i\sin nt)\\
&=\sum_{n=1}^{\infty} \frac{\pi}{ni}(-\cos nt+i\sin nt)+\sum_{n=1}^{\infty} \frac{\pi}{ni}(\cos nt+i\sin nt)\\
&=\sum_{n=1}^{\infty} \frac{\pi}{ni}(-\cos nt+\cos nt +2i\sin nt)\\
&=\sum_{n=1}^{\infty} \frac{2\pi}{n}\sin nt
\end{align*}



\end{itemize}


\vspace{0.2 cm}

\end{itemize}





\end{itemize}


\end{document}

%------------------------------------------------------------------------------
% End of journal.tex
%------------------------------------------------------------------------------
