\documentclass[12pt, reqno]{amsart}

\usepackage[margin=3.5 cm]{geometry}
\usepackage{graphicx, mathabx}
\usepackage{color}
\newtheorem{theorem}{Theorem}[section]
\newtheorem*{theorem*}{Theorem}          %theorem without number
\newtheorem{prop}{Proposition}[section]
\newtheorem{coro}{Corollary}[section]
\newtheorem{lemma}[theorem]{Lemma}
\newtheorem{conj}{Conjecture}[section]
\newtheorem{obs}{Observation}[section]

\theoremstyle{definition}
\newtheorem{definition}[theorem]{Definition}
\newtheorem{example}[theorem]{Example}
\newtheorem{xca}[theorem]{Exercise}

\theoremstyle{remark}
\newtheorem{remark}[theorem]{Remark}

\newcommand{\abs}[1]{\lvert#1\rvert}
\newcommand{\norm}[1]{\lVert#1\rVert}
\DeclareMathOperator{\re}{Re}
\DeclareMathOperator{\im}{Im}
\newcommand{\ud}{\mathrm{d}}

\begin{document}

\title[Math 357 - Harmonic Analysis]{Math 357 - Peer feedback for set 2}   


% rename the file: LastName_FirstName_feedback_setXXX (replace XXX with correct set number)


\maketitle

\section*{}


Multiple computations on Set 2 involved periodic functions. We can often shorten computations by using periodicity. Sometimes, this may avoid a computation through a simple geometric argument. For example, on problem 5, integration by parts gave the term \begin{equation}
  f(t)\overline{e_n(t)}\bigg\rvert_0^1
\end{equation}Noting that both $f$ and $\overline{e_n}$ are 1-periodic, we see their product is 1-periodic. So, the difference in boundaries of (1) reduces to 0.


On problem 6, we computed the real Fourier series of the sawtooth function. Many noticed that $\phi$ was $2\pi$-periodic and odd. That it has both of these properties makes it's Fourier series have only sine terms: $a_n=0$ for all $n$. \begin{align*}
\hat \phi_n&=\int_0^1 \phi(t)\overline{e_n(t)}\ dt\\
&=\int_{-\frac{1}{2}}^{\frac{1}{2}}\phi(t)\overline{e_n(t)}\ dt
\end{align*}
The real part of this integral is the product of an even fuction ($\cos$) with the odd $\phi$, so it reduces to 0. Computing the Fourier coefficients from \begin{align}
  a_n&=\re(\hat\phi_n)\\
  b_n&=\im(\hat\phi_n)
\end{align}or directly from the their definitions gives the same result. These instances show how properties of odd and even, and periodic functions can simplify or avoid unecessary calculations. 



\end{document}

%------------------------------------------------------------------------------
% End of journal.tex
%------------------------------------------------------------------------------
