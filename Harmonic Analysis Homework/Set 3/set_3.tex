\documentclass[12pt, reqno]{amsart}

%\usepackage{upgreek}
\usepackage[margin=3.5 cm]{geometry}
\usepackage{graphicx, mathabx}
\usepackage{color}
%\usepackage{subfigure}
\newtheorem{theorem}{Theorem}[section]
\newtheorem*{theorem*}{Theorem}          %theorem without number
\newtheorem{prop}{Proposition}[section]
\newtheorem*{prop*}{Proposition}  % proposition without number
\newtheorem{coro}{Corollary}[section]
\newtheorem{lemma}[theorem]{Lemma}
\newtheorem{conj}{Conjecture}[section]
\newtheorem{obs}{Observation}[section]

\theoremstyle{definition}
\newtheorem{definition}[theorem]{Definition}
\newtheorem{example}[theorem]{Example}
\newtheorem{xca}[theorem]{Exercise}

\theoremstyle{remark}
\newtheorem{remark}[theorem]{Remark}

%\numberwithin{equation}

%    Absolute value notation
\newcommand{\abs}[1]{\lvert#1\rvert}
\newcommand{\norm}[1]{\lVert#1\rVert}
\DeclareMathOperator{\re}{Re}
\DeclareMathOperator{\im}{Im}
\newcommand{\ud}{\mathrm{d}}

\begin{document}

\title[Math 357 - Harmonic Analysis]{Problem set no. 3 - Isaac Viviano}

\begin{titlepage}
   \maketitle

   I affirm that I have adhered to the Honor Code on this assignment. 
   \begin{flushright} Isaac Viviano \end{flushright}
\end{titlepage}

\section*{}

\section{Solutions:} 

\begin{itemize}

\item {\bf{Problem 1 - Interchanging integrals and derivatives:}}

\vspace{0.1 cm}
\begin{theorem*}
Let $f: [a,b] \times [c,d] \to \mathbb{R}$ be a continuous. If $\frac{\partial f}{ \partial y}$ exists and and is continuous on all of $[a,b] \times [c,d]$, then one has
\begin{equation*}
\dfrac{\ud}{\ud y} \int_a^b f(x,y) ~\ud x = \int_a^b \frac{\partial f}{\partial y}(x,y) ~\ud x ~\mbox{.}
\end{equation*}
\end{theorem*}

\begin{proof}
   
   Let $\epsilon>0$ be given. Since $\frac{\partial f}{\partial y}$ is continuous, $\frac{\partial f}{\partial y}:[c,d]\rightarrow \mathbb{R}$ is uniformly continuous. Pick $\delta>0$ such that for all $x\in[a,b]$ and all $y_{0},y_{1}\in[c,d]$, $$|y_{0}-y_{1}|<\delta\implies \left|\frac{\partial f}{\partial y}(x,y_{0})- \frac{\partial f}{\partial y}(x,y_{1})\right|$$Fix $x\in[a,b]$. At $y_{0}$,
   \begin{align*}
   \frac{d}{dy}\int_{a}^{b}f(x,y)&:=  \frac{d}{dy}F(y_{0})\\
   &\ = \lim_{y\rightarrow y_{0}}\frac{F(y)-F(y_{0})}{y-y_{0}}\\
   \end{align*}
   Now,
   \begin{align*}
   \left|\frac{F(y)-F(y_{0})}{y-y_{0}}-\int_{a}^{b} \frac{\partial  f}{\partial y}(x,y_{0})\ dx\right|&= \left|\frac{\int_{a}^{b}f(x,y)\ dx-\int_{a}^{b}f(x,y_{0})}{y-y_{0}}\ dx-\int_{a}^{b} \frac{\partial  f}{\partial y}(x,y_{0})\ dx\right|\\
   &= \left|\int_{a}^{b}\frac{f(x,y)-f(x,y_{0})}{y-y_{0}}\ dx-\int_{a}^{b} \frac{\partial  f}{\partial y}(x,y_{0})\ dx\right|\\
   &= \left|\int_{a}^{b} \frac{f(x,y)-f(x,y_{0})}{y-y_{0}}-\frac{\partial  f}{\partial y}(x,y_{0})\ dx\right|\\
   &\le\int_{a}^{b}\left|\frac{f(x,y)-f(x,y_{0})}{y-y_{0}}- \frac{\partial  f}{\partial y}(x,y_{0})\right|\ dx\\
   \end{align*}
   By the (MVT), pick $y_{1}\in[y,y_{0}]$ with $$\frac{f(x,y)-f(x,y_{0})}{y-y_{0}}= \frac{\partial f}{\partial y}(x,y_{1})$$Since $|y_{0}-y|<\delta$ and $y_{1}\in[y,y_{0}]$, $|y_{0}-y_{1}|\le|y_{0}-y|<\delta$ \begin{align*}
   \int_{a}^{b}\left|\frac{f(x,y)-f(x,y_{0})}{y-y_{0}}- \frac{\partial  f}{\partial y}(x,y_{0})\right|\ dx&= \int_{a}^{b} \left|\frac{\partial f}{\partial y}(x,y_{1})-\frac{\partial f}{\partial x}(x,y_{0})\right|\ dx\\
   &\le \int_{a}^{b} \frac{\epsilon}{b-a}\ dx\\
   &= \epsilon
   \end{align*}
   
\end{proof}

\vspace{0.2 cm}
\item {\bf{Problem 2 - Properties of convolutions:}} 


\vspace{0.1 cm}
\begin{itemize}

\vspace{0.1 cm}
\item[(c)] {\bf{Smoothening property}}, i.e. 
   \begin{itemize}
      \item[(c-i)] 
      
      \begin{prop}
         
      If $f \in \mathcal{R}_1(\mathbb{R}; \mathbb{C})$ and $g \in \mathcal{C}_1(\mathbb{R}; \mathbb{C})$, then $f \star g \in \mathcal{C}_1(\mathbb{R}; \mathbb{C})$.
      \end{prop}
      

\begin{proof}
   Suppose $f\in \mathcal{R}_{1}(\mathbb{R};\mathbb{C})$ and $g\in \mathcal{C}_{1}(\mathbb{R};\mathbb{C})$. Let $$h(t):=(f\star g)(t)=\int_{0}^{1}f(s)g(t-s)\ ds$$
Note that $f$ is Riemann integrable and thus bounded, and $g$ is periodic and continuous and thus uniformly continuous.

Let $\epsilon>0$ be given and fix $t_{0}\in \mathbb{R}$. Since $g$ is uniformly continuous, pick $\delta>0$ such that for all $x,y\in \mathbb{R}$, if $|x-y|<\delta$, $|g(x)-g(y)|< \frac{\epsilon}{\|f\|_{\infty}}$. Let $t\in \mathbb{R}$ with $|t-t_{0}|<\delta$. Then for all $s\in[0,1]$, \begin{align*}|(t-s)-(t_{0}-s)|&=|t-s-t_{0}+s|\\&=|t-t_{0}|\\&<\delta\end{align*}So, 
\begin{align*}
|h(t)-h(t_{0})|&= \left|\int_{0}^{1}f(s)g(t-s)\ ds-\int_{0}^{1}f(s)g(t_{0}-s)\ ds\right|\\
&= \left|\int_{0}^{1}f(s)g(t-s)- f(s)g(t_{0}-s)\ ds\right|\\
&= \left|\int_{0}^{1}f(s)(g(t-s)- g(t_{0}-s))\ ds\right|\\
&\le \int_{0}^{1}\left|f(s)\right|\cdot\left|g(t-s)-g(t_{0}-s)\right|\ ds\\
&< \int_{0}^{1}|f(s)|\cdot \frac{\epsilon}{\|f\|_{\infty}}\ ds\\
	&= \frac{\epsilon}{\|f\|_{\infty}}\int_{0}^{1}\left|f(s)\right|\ ds\\
&\le \frac{\epsilon}{\|f\|_{\infty}}\cdot\|f\|_{\infty}\quad(\text{ML estimate})\\
&= \epsilon
\end{align*}
Therefore, $h$ is continuous.

For all $t$, we have
\begin{align*}
h(t)&= \int_{0}^{1}f(s)g(t-s)\ ds\\
&= \int_{0}^{1}f(s)g(t-s+1)\ ds\quad(g\text{ is }1\text{-periodic})\\
&= \int_{0}^{1}f(s)g((t+1)-s)\ ds\\
&= h(t+1)
\end{align*}showing that $h$ is also $1$-periodic.
\end{proof}
      
      \item[(c-ii)] 
      \begin{prop}
         
      If $f \in \mathcal{C}_1(\mathbb{R}; \mathbb{C})$ and, for some $k \in \mathbb{N}$, one has $g \in \mathcal{C}_1^k(\mathbb{R}; \mathbb{C})$, then $f \star g \in \mathcal{C}_1^k(\mathbb{R}; \mathbb{C})$. 
      \end{prop}

\begin{proof}
   Let $f\in\mathcal{C}_{1}(\mathbb{R};\mathbb{C})$ and $g\in \mathcal{C}^{k}_{1}(\mathbb{R};\mathbb{C})$. Let $h=f\star g$. We have

Note that $f(s)g(t-s)$ is $k$-times continuously differentiable with respect to $t$. By problem (1), \begin{align*}
\frac{d }{d t}h(t)&= \int_{0}^{1}\frac{\partial }{\partial t}(f(s)g(t-s))\ ds\\
&= \int_{0}^{1} f(s)g'(t-s)\ ds
\end{align*}
We may do this $k$ times iteratively, showing $$h^{(k)}(t)=\int_{0}^{1}f(s)g^{(k)}(t-s)\ ds$$Since $f$ and $g^{(k)}$ are continuous and $1$-periodic, $h^{(k)}$ is continuous and $1$-periodic. Therefore, $$f\star g=h\in\mathcal{C}_{1}^{k}(\mathbb{R};\mathbb{C})$$ 

\end{proof}
      
   \end{itemize}
   
\vspace{0.1 cm}
\item[(d)] {\bf{A-priori bound for convolutions}} 
\begin{prop}
   
If  $f,g \in \mathcal{R}_1(\mathbb{R}; \mathbb{C})$, then $f \star g$ is bounded with
\begin{equation} \label{eq_conv_bound}
\Vert f \star g \Vert_\infty \leq \min\{ \Vert f \Vert_\infty \cdot \Vert g \Vert_1 ~;~ \Vert g \Vert_\infty \cdot \Vert f \Vert_1 \} 
\end{equation}
\end{prop}

\begin{proof}
   It suffices to show that $$\|f\star g\|_{\infty}\le\|g\|_{\infty}\cdot\|f\|_{1}$$Then, the symmetry property in part (a) gives \begin{align*}
      \|f\star g\|_{\infty}&= \|g\star f\|_\infty\le \|f\|_{\infty}\cdot\|g\|_{1}
      \end{align*}
      Proof: \begin{align*}
      \|f\star g\|_{\infty}&= \left\|\int_{0}^{1}f(s)g(t-s)\ ds\right\|_{\infty}\\
      &= \sup_{t\in\mathbb{R}} \left\{\left|\int_{0}^{1}f(s)g(t-s)\ ds\right|\right\}\\
      &\le \sup_{t\in \mathbb{R}}\left\{\int_{0}^{1}|f(s)|\cdot|g(t-s)|\ ds\right\}\\
      &\le \sup_{t\in \mathbb{R}}\left\{\|g\|_{\infty}\int_{0}^{1}|f(s)|\ ds\right\}\\
      &= \|g\|_{\infty}\int_{0}^{1}|f(s)|\ ds\\
      &= \|g\|_{\infty}\cdot\|f\|_{1}
      \end{align*}
\end{proof}

\vspace{0.1 cm}
\item[(e)] {\bf{Continuity w.r.t. uniform convergence}}
\begin{prop}
   
if $f \in \mathcal{R}_1(\mathbb{R}; \mathbb{C})$ and $(g_n)$ is a sequence in $\mathcal{R}_1(\mathbb{R}; \mathbb{C})$ such that $g_n \to g$ in the $L^1$-sense, then 
\begin{equation}
   f \star g_n \to f \star g ~\mbox{, uniformly .}
\end{equation}
\end{prop}

\begin{proof}
   Suppose $g_{n}\rightarrow g$ with respect to the L1 norm and let $\epsilon>0$ be given. Pick $N\in \mathbb{N}$ such that for all $n\ge N$, $$\|g_{n}-g\|_{1}< \frac{\epsilon}{\|f\|_{\infty}}$$Then, 
\begin{align*}
\|f\star g_{n}-f\star g\|_{\infty}&= \|f\star(g_{n}-g)\|_{\infty}\\
&\le \min\{\|f\|_{\infty}\cdot \|g_{n}-g\|_{1}, \|g_{n}-g\|_{\infty}\cdot\|f\|_{1}\}\\
&\le \|f\|_{\infty}\cdot \|g_{n}-g\|_{1}\\
&< \|f\|_{\infty}\cdot \frac{\epsilon}{\|f\|_{\infty}}\\
&= \epsilon
\end{align*}So, $f\star g_{n}$ converges to $f\star g$ uniformly.
\end{proof}

\end{itemize}

\vspace{0.2 cm}
\item {\bf{Problem 3 - ``the Dirichlet kernel is not a periodic approximate identity:''}} Recall the Dirichlet kernel from class: 
\begin{equation*}
D_n(x) = \dfrac{\sin(\pi (2 n + 1)x)}{\sin(\pi x)} ~\mbox{, $x \in [-\frac{1}{2}, \frac{1}{2}]$, for $n \in \mathbb{N}$.}
\end{equation*}
In the following you will show that 
\begin{equation} \label{eq_diri}
\frac{4}{\pi^2} \log(n) \leq \int_{-1/2}^{1/2} \vert D_n(x) \vert ~\ud x \leq 3 + \log(n) ~\mbox{,}
\end{equation}
whence $(D_n)$ is {\em{not}} an approximate identity because it fails the property (PAI-2) from our definition. To prove (\ref{eq_diri}), follow the outline below.

\vspace{0.1 cm}
\begin{itemize}
\item[(a)]
\begin{prop}
If $f: [1, +\infty] \to \mathbb{R}$ is non-negative and monotone decreasing then
\begin{equation*}
\int_1^{N+1} f(x) ~\ud x \leq \sum_{n=1}^{N} f(n) \leq f(1) + \int_1^N f(x) ~\ud x ~\mbox{.}
\end{equation*}
\end{prop}

\begin{proof}
   Note that for $a,b\ge 1$, $$\sup\{f(x):x\in[a,b]\}=f(a)$$and $$\inf\{f(x):x\in [a,b]\}=f(b)$$because $f$ is monotone decreasing.

Thus, for any partition $s_{1},\ldots,s_{N+1}$ of $[1,N+1]$, our upper and lower bounds are given by \begin{align*}
M_{i}&= f(s_{i})\\
m_{i}&= f(s_{i+1})\\
\end{align*}Consider the partition given by $s_{i}=i$ for all $1\le i\le N+1$. By definition, \begin{align*}
\int_{1}^{N+1}f(x)\ dx&\le \sum_{i=1}^{N}M_{i}(s_{i+1}-s_{i})\\
&= \sum_{i=1}^{N}f(s_{i})(i+1-i)\\
&= \sum_{i=1}^{N}f(i)
\end{align*}and \begin{align*}
f(1)+\int_{1}^{N}f(x)\ dx&\ge f(1)+\sum_{i=1}^{N-1}m_{i}(s_{i+1}-s_{i})\\
&= f(1)+\sum_{i=1}^{N-1}f(s_{i+1})(i+1-i)\\
&= f(1)+\sum_{i=1}^{N-1}f(i+1)\\
&= f(1)+\sum_{i=2}^{N}f(i)\quad(\text{change index})\\
&= \sum_{i=1}^{N}f(i)
\end{align*}
\end{proof}

\vspace{0.1 cm}
\item[(b)]

\begin{prop}
   \begin{equation} 
      \frac{4}{\pi^2} \log(n) \leq \int_{-1/2}^{1/2} \vert D_n(x) \vert ~\ud x \leq 3 + \log(n) ~\mbox{,}
      \end{equation}
   
\end{prop}

\begin{proof}
   First note that $|D_{n}(x)|$ is even:

\begin{align*}
|D_{n}(-x)|&= \left|\frac{\sin(\pi(2n+1)(-x))}{\sin(-\pi x)}\right|\\
&= \left|\frac{-\sin(\pi(2n+1)x)}{-\sin(\pi x)}\right|\\
&= \left|\frac{\sin (\pi(2n+1)x)}{\sin(\pi x)}\right|\\
&= |D_{n}(x)|
\end{align*}Therefore, \begin{align*}
\int_{- \frac{1}{2}}^{\frac{1}{2}}|D_{n}(x)|\ dx&= 2\int_{0}^{\frac{1}{2}}\left|D_{n}(x)\right|\ dx:=2I_{n}
\end{align*}
Let $$4a_{n}= \frac{2}{2n+1}$$be the period of $f_{n}:=|\sin(\pi(2n+1)x)|$. $f_n$ has zeros when $\pi(2n+1)x$ is an integer multiple of $\pi$, that is, $(2n+1)x$ is an integer, or $x=2ka_n$ for some $k\in \mathbb{N}$. It has maxima at $\pi(2n+1)x- \frac{\pi}{2}$ is an integer. These occur when $x=(2k+1)a_n$ for some $k\in \mathbb{N}$. The only extrema are these zeros and maxima. Since $f_{n}$ is continuous, it is monotonic non-decreasing from a zero to a maximum and monotonic non-increasing from a maximum to a zero. These intervals of monotonicity are \begin{align*}
[0,a_{n}]\\
[a_{n},2a_{n}]\\
[2a_{n},3a_{n}]\\
[3a_{n},4a_{n}]
\end{align*}on its period interval $[0,4a_n ]$. 
\begin{align*}
I_{n}&= \int_{0}^{\frac{1}{2}}|D_{n}(x)|\ dx\\
&= \int_{0}^{\frac{1}{2}}\frac{|\sin (\pi(2n+1)x)|}{|\sin(\pi x)|}\ dx\\
&= \sum_{j=0}^{2n}\int_{ja_{n}}^{(j+1)a_{n}} \frac{|\sin(\pi(2n+1)x)|}{\sin \pi x}\ dx\\
&\ge \sum_{j=0}^{2n}\int_{ja_{n}}^{(j+1)a_{n}} \frac{|\sin(\pi(2n+1)x)|}{\pi x}\ dx\\
&\ge \sum_{j=0}^{2n} \frac{1}{\pi(j+1)a_{n}}\int_{ja_{n}}^{(j+1)a_{n}} |\sin(\pi(2n+1)x)|\ dx\\
\end{align*}
From the previous observation, \begin{align*}
\int_{ja_{n}}^{(j+1)a_{n}}|\sin(\pi(2n+1)x)\ dx&= \int_{0}^{a_{n}}\sin(\pi(2n+1)x)\ dx
\end{align*}

We may use the change of variables $u= \pi(2n+1)x$ to write \begin{align*}
\int_{ja_{n}}^{(j+1)a_{n}}\sin(\pi(2n+1)x)&= \int_{0}^{a_{n}}\sin(\pi(2n+1)x)\ dx\\
&= \int_{0}^{\frac{\pi}{2}} \frac{1}{\pi(2n+1)}\sin u\ du\\
&= \frac{-\cos u}{\pi(2n+1)}\bigg|_{0}^{\frac{\pi}{2}}\\
&= 0 - \frac{-1}{\pi(2n+1)}\\
&= \frac{2a_{n}}{\pi}
\end{align*}
Thus, \begin{align*}
I_{n}&\ge \sum_{j=0}^{2n} \frac{1}{\pi(j+1)a_{n}}\cdot \frac{2a_{n}}{\pi}\\
&= \sum_{j=1}^{2n+1} \frac{2}{\pi^{2}j}\\
&= \frac{2}{\pi^{2}}\sum_{j=1}^{2n+1} \frac{1}{j}\\
&\ge \frac{2}{\pi^{2}}\int_{1}^{2n+2} \frac{1}{x}\ dx\quad\quad(\text{a})\\
&= \frac{2}{\pi^{2}}\ln x\bigg|^{2n+2}_{1}\\
&= \frac{2}{\pi^{2}}\ln(2n+2)\\
&\ge \frac{2}{\pi^{2}}\ln n
\end{align*}
For the upper bound,

\begin{align*}
I_{n} &= \int_{0}^{\frac{1}{2}}\frac{|\sin (\pi(2n+1)x)|}{|\sin(\pi x)|}\ dx\\
&= \int_{0}^{a_{n}}\frac{|\sin (\pi(2n+1)x)|}{|\sin(\pi x)|}\ dx+\sum_{j=1}^{2n}\int_{ja_{n}}^{(j+1)a_{n}} \frac{|\sin (\pi(2n+1)x)|}{|\sin(\pi x)|}\ dx\\
&= \int_{0}^{a_{n}}\left|\frac{\sin(\pi(2n+1)x)}{\pi(2n+1)x}\right| \cdot\left| \frac{\pi(2n+1)x}{\pi x}\right|\cdot \left| \frac{\pi x}{\sin(\pi x)}\right|\ dx+\\
&\quad\quad\quad\sum_{j=1}^{2n}\int_{ja_{n}}^{(j+1)a_{n}} \frac{|\sin (\pi(2n+1)x)|}{|\sin(\pi x)|}\ dx\\
&\le \int_{0}^{a_{n}}2n+1\ dx+\sum_{j=1}^{2n}\int_{ja_{n}}^{(j+1)a_{n}} \frac{|\sin (\pi(2n+1)x)|}{|\sin(\pi x)|}\ dx\\
&\le \int_{0}^{\frac{1}{2(2n+1)}}2n+1 \ dx+\sum_{j=1}^{2n}\int_{ja_{n}}^{(j+1)a_{n}} \frac{1}{2x}\ dx\\
&= \frac{1}{2}+\sum_{j=1}^{2n} \frac{1}{2ja_{n}}\cdot a_{n}\quad(\text{ML-estimate})\\
&= \frac{1}{2}+ \frac{1}{2}\sum_{j=1}^{2n} \frac{1}{j}\\
&\le \frac{1}{2}+ \frac{1}{2}\left(1+\int_{1}^{2n} \frac{1}{x}\ dx\right)\\
&= \frac{1}{2}+ \frac{1}{2}\left( \ln x\bigg|_{1}^{2n}\right)\\
&= \frac{1}{2}+ \frac{1}{2}(\ln 2n)
\end{align*}

We have \begin{align*}
I_{n}&\ge \frac{2}{\pi^{2}}\log n\\
I_{n}&\le \frac{1}{2}+ \frac{1}{2}(\ln 2n)\\
2I_{n}&\le 1+\ln (2n)\\
&= 1+\ln 2+\ln n\\
&\le 3+\ln n\\
2I_{n}&\ge \frac{4}{\pi^{2}}\log n
\end{align*}
showing the desired bounds.
\end{proof}

\end{itemize}

\vspace{0.2 cm}
\item {\bf{Problem 4 - Existence of Minimal Periods for Periodic Functions:}} 


\begin{lemma}
   Let $f:\mathbb{R}\rightarrow \mathbb{C}$ be periodic and let $\Pi_f$ denote the set of all periods of $f$. Then, 
   \begin{enumerate}
   \item $\Pi_f\cup\{0\}$ is an additive subgroup of $\mathbb{R}$
   \item If $0\notin \text{cl }\Pi_f$, then $\Pi_f$ has no other limit point and $f$ has a minimal period
\end{enumerate}
\end{lemma}

\begin{proof}
   
   (1) Let $a,b\in \Pi_{f}\cup\{0\}$. We have that for all $x\in \mathbb{R}$, $$f(x)=f(a+x)$$and $$f(x)=f(b+x)=f(b+(x-2b))=f(x-b)$$

Then, \begin{align*}
f(x+a-b)&= f((x+a)-b)\\
&= f(x+a)\\
&= f(x)
\end{align*}
So, $a-b\in \Pi_f\cup\{0\}$, showing that $\Pi_f\cup\{0\}$ is an additive subgroup of $\mathbb{R}$. 

(2) For contradiction, suppose $0\notin \text{cl }\Pi_{f}$ and $\Pi_f$ has a limit point. Let $a\in \text{cl }\Pi_{f}$ be a limit point of $\Pi_{f}$. Then, there exists a sequence $$\{a_{n}\}\subseteq \Pi_f$$with $a_{n}\rightarrow a$. Let $\epsilon>0$ be given. Pick $N$ such that for all $n,m\ge N$, $$|a_{n}-a_{m}|<\epsilon$$Fix $n\ge N$. We have $a_{n}-a_{m},a_{m}-a_{n}\in \Pi_{f}$ by (1). Let $$b_{n}=\max\{a_{n}-a_{m},a_{m}-a_{n}\}=|a_{n}-a_{m}|$$So, $0\le b_{n}<\epsilon$. Therefore, $b_{n}\in \Pi_{f}$ and $b_{n}\rightarrow 0$. This shows that $0$ is a limit point of $\Pi_{f}$ and thus $0\in \text{cl }\Pi_{f}$.

Now suppose $0\notin \text{cl }\Pi_f$. Let $T$ be a period of $f$. If $$[0,T]\cap \Pi_{f}$$is infinite, it has a limit point. But, this would be a limit point of $\Pi_f$, which has no limit points. Thus, $[0,T]\cap \Pi_f$ is finite and nonempty. $[0,T]\cap \Pi_f$ must have a minimum, showing that $f$ has a minimal positive period. 
   
\end{proof}


\begin{prop}
   If $f:\mathbb{R}\to\mathbb{C}$ is periodic and has atleast one point of continuity, then $f$ has a minimal period.
\end{prop}

\begin{proof}
   
Suppose $f:\mathbb{R}\rightarrow \mathbb{C}$ is periodic, non-constant, and has at least point of continuity. 

Let $$P=\{|p|:f \text{ is }p \text{-periodic}\}$$
Note that $P$ is nonempty since $f$ is periodic and bounded below by zero, so $P$ has an infimum $\alpha=\inf P$



Let $x_{0}$ be a point of continuity of $f$. Since $f$ is not constant, pick $y\in \mathbb{R}$ with $f(x_{0})\ne f(y)$. Pick $\delta>0$ such that for all $x\in \mathbb{R}$ with $|x_{0}-x|<\delta$, $$|f(x_{0})-f(x)|<|f(x_{0})-f(y)|$$ 
Claim: $\frac{\delta}{2}$ is a lower bound for $P$. If $p\le \frac{\delta}{2}$ is a period of $f$, then for all $n\in \mathbb{Z}$, $$f(y+np)=f(y)$$Taking $$n=\left\lfloor \frac{1}{p}(x_{0}-y)\right\rfloor$$we have \begin{align*}
y+np &= y+p\left\lfloor \frac{1}{p}(x_{0}-y)\right\rfloor\\
&\le y+p\cdot \frac{1}{p}(x_{0}-y)\\
&= y+x_{0}-y\\
&= x_{0}\\
&< x_{0}+\delta
\end{align*}
and \begin{align*}
y+np&= y+p\left\lfloor \frac{1}{p}(x_{0}-y)\right\rfloor\\
&\ge y+p\left(\frac{1}{p}(x_{0}-y)-1\right)\\
&= y+x_{0}-y-p\\
&= x_{0}-p\\
&\ge x_{0}- \frac{\delta}{2}\\
&> x_{0}-\delta
\end{align*}
Therefore, $$
y+np\in(x_{0}-\delta,x_{0}+\delta)
$$since $$y+np< x_{0}+\delta$$and $$y+np> x_{0}-\delta$$
But $f(y+np)\in(x_{0}-\delta,x_{0}+\delta)$ implies that $$
|f(x_{0})-f(y+np)|<|f(x_{0})-f(y)|
$$In particular, $$|f(x_{0})-f(y+np)|\ne|f(x_{0})-f(y)|$$Thus, $$f(y+np)\ne f(y)$$contradicting that $p$ is a period. By contradiction, for all $p\in P$, $\frac{\delta}{2}<p$.


Since $\alpha$ is greater than or equal to any lower bound for $P$, $$\alpha\ge \frac{\delta}{2}>0$$Suppose 0 were a limit point of $\Pi_f$. Then we would have a sequence $p_{n}$ in $\Pi_f$ with $\pi_{n}\rightarrow 0$. But $|p_{n}|$ is a sequence in $P$ and $|p_{n}|\rightarrow 0$, contradicting that $\alpha\ge0$. Therefore, $0$ is a not a limit point of $\Pi_f$, and $f$ is has a minimal positive period by Lemma 1.1.





\end{proof}



\end{itemize}

\end{document}

%------------------------------------------------------------------------------
% End of journal.tex
%------------------------------------------------------------------------------
