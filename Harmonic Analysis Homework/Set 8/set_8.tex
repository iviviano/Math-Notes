\documentclass[12pt, reqno]{amsart}

%\usepackage{upgreek}
\usepackage[margin=3.5 cm]{geometry}
\usepackage{graphicx,mathabx,hyperref}
\usepackage{color}
\usepackage{float}
%\usepackage{subfigure}
\newtheorem{theorem}{Theorem}[section]
\newtheorem*{theorem*}{Theorem}          %theorem without number
\newtheorem{prop}{Proposition}[section]
\newtheorem*{prop*}{Proposition}  % proposition without number
\newtheorem{coro}{Corollary}[section]
\newtheorem{lemma}[theorem]{Lemma}
\newtheorem{conj}{Conjecture}[section]
\newtheorem{obs}{Observation}[section]

\theoremstyle{definition}
\newtheorem{definition}[theorem]{Definition}
\newtheorem{example}[theorem]{Example}
\newtheorem{xca}[theorem]{Exercise}

\theoremstyle{remark}
\newtheorem{remark}[theorem]{Remark}

%\numberwithin{equation}

%    Absolute value notation
\newcommand{\abs}[1]{\lvert#1\rvert}
\newcommand{\norm}[1]{\lVert#1\rVert}
\DeclareMathOperator{\re}{Re}
\DeclareMathOperator{\im}{Im}
\newcommand{\ud}{\mathrm{d}}

\begin{document}

\title[Math 357 - Harmonic Analysis]{Problem set no. 8 - Isaac Viviano}

\begin{titlepage}
    
\maketitle

I affirm that I have adhered to the Honor Code in this assignment. Isaac Viviano

\end{titlepage}

blank
\newpage


\section*{}

% {\bf{Only the problems marked as ``{\em{mandatory problems}}'' are to be turned in as part of this week's problem set.}} Any remaining problems are {\em{recommended}}, but you do not need to turn them in. 

% For the recommended problems, I suggest to just think about a strategy, possibly jotting down some rough ideas, but only work out the details if you have extra time at your disposal. {\bf{As stated in the syllabus, you are allowed to use the results of recommended problems (even if you do not prove them) as well as any of the problems from previous sets.}} Just make sure to clearly reference them in your work. 

% \vspace{0.2 cm}
% {\underline{\bf{Mandatory problems:}}} all

% \vspace{0.1 cm}
% {\bf{From the above mandatory problems, you have to turn in all mandatory problems written up using TeX.}} 
% \vspace{0.1 cm}

% \vspace{0.2 cm}
% {\underline{\bf{Recommended problems:}}} $\emptyset$

% \vspace{0.2 cm}
% {\underline{NOTE:}} Even though you do not have to do the recommended problems, you are explicitly allowed ({\tiny{and should consider to}}) use their results in your work; as with all other results you are using, just make it clear by referencing appropriately what you use; see also the section ``homework'' in the syllabus.

% \vspace{0.2 cm}
% In the following, I will use [CM] for my book manuscript and [DJB] for David J. Benson's text.

\section{Problems:} 

\begin{itemize}

\item {\bf{Problem 1 - Existence of Diophantine Numbers:}} %You showed in set 7, problem 2(c), that [CM] Proposition 7.5.2 implies the existence of Diophantine numbers. Use the strategy outlined after Remark 7.9 in [CM] to prove this proposition.

\begin{prop}
    Consider the cohomological equation
    \begin{equation}\label{cohom}
        h(x+ \alpha)-h(x)=f(x)-\hat f_{0}
    \end{equation}
    
    Let $\phi:(0,\infty)\rightarrow (0,\infty)$ be a strictly increasing function which satisfies $$\sum_{n=1}^{\infty} \frac{1}{\phi(n)}<+\infty$$Then, there exists a 1-periodic, continuous function $f\in\mathcal{C}_{1}(\mathbb{R};\mathbb{R})$ with $\hat f_{0}=0$ and $$\hat f_{\pm n}= \frac{1}{\phi(n)},\text{ for all }n\in \mathbb{N}$$and an irrational $\alpha\in \mathbb{R}$ so that the cohomological equation (\ref{cohom}) has no continuous solution $h$.
\end{prop}

\begin{proof}
    Since $\phi$ maps into $(0,\infty)$, $$\left\|\frac{1}{\phi(n)}e_{\pm n}\right\|_{\infty}= \left\| \frac{1}{\phi(n)}\right\|_{\infty}= \frac{1}{\phi(n)}$$
Since $\frac{1}{\phi(n)}$ is summable, the Weierstrass $M$-test implies that the Fourier series $$\sum_{n=-\infty}^{\infty} \frac{1}{\phi(|n|)}e_{n}$$converges uniformly. Defining $$f:=\sum_{n=-\infty}^{\infty} \frac{1}{\phi(|n|)}e_{n}$$
we see that $$\hat f_{\pm n}= \frac{1}{\phi(n)}$$



Define two mutually recursive sequences $\{n_{k}\}_{k\in\mathbb{N}}$ and $\{b_{k}\}_{k\in \mathbb{N}}$: 
\begin{align*}
b_{1}&:= 1\\
n_{k}&:= 10^{b_{k}}\\
b_{k+1}&:= \min\left(\{l\in \mathbb{N}: \frac{4\pi}{10^{l}}\le \frac{1}{\phi(n_{k})}\}\setminus \{b_{1},\ldots,b_{k-1}\}\right)
\end{align*}
For each $k\in \mathbb{N}$, let 
\begin{align*}
&x_{b_{k}}:= 1\\
&x_{b_{k}+1},\ldots,x_{b_{k+1}-1}:=0
\end{align*}
Define 
\begin{equation} \label{alpha}
\alpha:=\sum_{k=0}^{\infty} \frac{x_{k}}{10^{k}}
\end{equation}
We have for each $k$, $$\left| \frac{x_{k}}{10^{k}}\right|\le \left| \frac{1}{10^{k}}\right|$$Since $\frac{1}{10^{k}}$ is a geometric series with radius less than 1, it converges absolutely. Thus, the series in (\ref{alpha}) converges to $\alpha$ by the comparison test. Note also that we may write the decimal expansion of $\alpha$ as $$\alpha=x_{0}.x_{1}x_{2}\ldots$$
Noting that all of the $b_{k}$'s are unique, we see that $\alpha$ is irrational since it never terminates or repeats. 

We estimate
\begin{align*}
n_{k}\alpha\mod1&= 10^{b_{k}}\sum_{l=0}^{\infty} \frac{x_{l}}{10^{l}}\mod1\\
&= \left(\underbrace{\sum_{l=0}^{\infty}10^{b_{k}-l}x_{l}}_{\in \mathbb{N}} -\underbrace{\sum_{l= b_{k}+1}^{\infty} \frac{x_{l}}{10^{l-b_{k}}}}_{<1}\right)\mod1\\
&= \sum_{l=b_{k}+1}^{\infty} \frac{x_{l}}{10^{l-b_{k}}}\\
&= \sum_{l=b_{k}+1}^{b_{k+1}} \frac{x_{l}}{10^{l-b_{k}}}+\sum_{l=b_{k+1}+1}^{\infty} \frac{x_{l}}{10^{l-b_{k}}}\\
&\le \sum_{l=b_{k}+1}^{b_{k+1}} \frac{x_{l}}{10^{l-b_{k}}}+ \frac{1}{10^{b_{k}+1}}
\\&= \frac{2}{10^{b_{k+1}}}
\end{align*}

\begin{figure}[H]\label{geom_arg}
    \includegraphics*[width=5in]{Long image 2024-04-15 21.06.54.jpg}
\end{figure}


Noting that we can associate $e^{2\pi in_{k}\alpha}$ with the rotation angle $n_{k}\alpha$. As shown in Figure \ref{geom_arg}, we estimate $|1-e^{2\pi in_{k}\alpha}|$ by the rotation map, since the straight-line distance between the points $1$ and $e^{2\pi in_{k}\alpha}|$ will always be less than the distance around the circle (of circumference $2\pi$). Thus, 
\begin{align*}
|e^{2\pi in_{k} \alpha}-1|&\le  2\pi (n_{k} \alpha-1\mod1)\\
&= 2\pi(n_{k}\alpha\mod1)\\
&\le 2\pi\cdot \frac{2}{10^{b_{k+1}}}\\
&\le \frac{1}{\phi(n_{k})}
\end{align*}

If $h$ were a continuous solution to (\ref{cohom}), we have 
\[
\hat h_{n}:=
\begin{cases} \frac{\hat f_{n}}{e^{2\pi in \alpha}-1} & \text{ if }n\ne0\\
\text{const} & \text{ if }n=0
\end{cases}
\]
In particular, we have 
\begin{equation} \label{FC_lower}
|\hat h_{n_{k}}|= \frac{\left|\hat f_{n_{k}}\right|}{|1-e^{2\pi in_{k}\alpha}|}\ge 1,~\text{for all }k\in \mathbb{N}
\end{equation}
The Riemann Lebesgue lemma implies that $\hat h_{n}\rightarrow 0$ as $n \rightarrow \infty$. But any subsequence of a convergent subsequence converges to the same value. (\ref{FC_lower}) implies that $\hat h_{n_{k}}\not\to0$. This contradicts the Riemann-Lebesgue lemma, showing that there is no continuous solution $h$ to (\ref{cohom}) for the $\alpha$ chosen.

\end{proof}

\vspace{0.2 cm}

\item {\bf{Problem 2 - Pythagoras' Tuning Problem and Approximate Return Times:}} 

\vspace{0.1 cm}
\begin{itemize}
% \item[(a)] %In class we proved Theorem 8.5 item (ii) in [CM] which shows that the sequence of continued fraction approximants $(\frac{p_n}{q_n})$ of a given irrational number $\alpha$ are precisely the best rational appoximants of the second kind (BA-2) defined in [CM] Definition 8.1(ii). In view of irrational rotations on the unit circle, this in particular implies that the approximate return times can be determined by the sequence of denominators $(q_n)$ (Corollary 8.1.1 in [CM]). 

% Read again [CM] Section 8.4 which summarizes our discussions of the proof of Theorem 8.5 item (ii). Moreover, to remind yourself of the connection of Corollary 8.1.1 in [CM] to the Pythagorean tuning problem, read [CM] Section 8.2 which discusses some consequences of Corollary 8.1.1 in [CM] for musical scales and temperaments. {\em{There is nothing to turn in here.}}

\vspace{0.1 cm}
\item[(b)] %Corollary 8.1.1 in [CM] can be generalized as stated in [CM] Theorem 8.10. Use Theorem 8.5 item (ii) (or equivalently, Corollary 8.1.1 in [CM]) to prove this generalization.

% {\underline{Comment:}} Don't forget to argue why $q_{n_M}$ in ([CM]; 8.132) can be found and why it is unique. 

\begin{theorem}
For a fixed irrational $\alpha\in \mathbb{R}\setminus\mathbb{Q}$, let $\frac{p_{n}}{q_{n}}$ be the sequence of convergents in the continued fraction expansion of $\alpha$. Given $M\in \mathbb{N}$, let $n_{M}\in \mathbb{N}$ be the unique natural number such that $$q_{n_{M}}\le M< q_{n_{M}+1}$$Then, one has
\begin{equation} \label{approx_ret}
\|\alpha\cdot q_{n_{M}}\|=\min_{1\le k\le M}\|\alpha\cdot k\|
\end{equation}
in particular, for all $n\in \mathbb{N}$, $$\|\alpha\cdot q_{n}\|=\min_{1\le k<q_{n+1}}\|\alpha\cdot k\|$$
\end{theorem}

\begin{proof}
    Show that $n_M$ exists and is unique:

Let $M\in \mathbb{N}$ be given. Let $A=\{k\in \mathbb{N}_{0}:q_{k}\le M\}$

Note that $A\subseteq \mathbb{N}$ is nonempty (it contains $q_{0}=1$) and bounded (by $M$). Thus, it has a maximum:
$$n_{M}:=\max A$$
We have $n_{M}\in A$ and thus, $$q_{n_{M}}\le M$$
That $n_{M}$ is the maximum of $A$ implies $n_{M}+1\notin A$. So, $q_{n_{M}+1}\not\le M\iff M<q_{n_{M}}$. 

If $k<n_{M}$, $$q_{k+1}\le q_{n_{M}}\le M$$so, $$q_{k+1}\not>M$$
If $k>n_{M}$, $$q_{k}\ge q_{n_{M}+1}>M$$so, $$q_{k}\not\le M$$
So, we have shown the existence and uniqueness of $n_{M}$. 

From Corollary 8.1.1, $q_{n_{M}}$ and $q_{n_{M}+1}$ are approximate return times and for all $q_{n_{M}}< k<q_{n_{M}+1}$, $k$ is not an approximate return time. If (\ref{approx_ret}) were false, we would have 
$$\|\alpha\cdot q_{n_{M}}\|>\|\alpha\cdot k_{0}\|=\min_{1\le k\le M}\|\alpha\cdot k\|$$
for some $1\le k_{0}\le M$ with $k_{0}\ne q_{n_{M}}$. Consider two cases. If $k_{0}<q_{n_{M}}$, this contradicts that $q_{n_{M}}$ is BA-2. Otherwise, $k_{0}>q_{n_{M}}$. So, $$\|\alpha\cdot k_{0}\|=\min_{1\le k\le M}\|\alpha\cdot k\|=\min_{1\le k\le k_{0}}\|\alpha\cdot k\|$$
which implies that $k_{0}$ is an approximate return time. Since the sequence of denominators $q_{n}$ is monotonic and $q_{n_{M}}<k_{0}<q_{n_{M}+1}$, we have contradicted Corollary 8.1.1, so (\ref{approx_ret}) holds.

Fix $n\in \mathbb{N}$ and let $M=q_{n+1}-1$. Note that taking $n_{M}=n$, we have $$q_{n_{M}}\le q_{n_{M}+1}-1=M$$since the $n_{M}$ satisfying this is unique for a given $M$, we have 
$$\|\alpha\cdot q_{n_{M}}\|=\min_{1\le k\le M}\|\alpha\cdot k\|$$which we may rewrite as $$\|\alpha\cdot q_{n}\|=\min_{1\le k< q_{n+1}}\|\alpha\cdot k\|$$
\end{proof}



\end{itemize}

\vspace{0.2 cm}

\item {\bf{Problem 3 - $\mathcal{C}^\infty$-Bump Functions:}} %This is a continuation of problem 4 from set 7 in which you introduced the class of Schwartz functions. Aside from the Gaussian (problem 4(b), set 7) other important examples of Schwartz functions are $\mathcal{C}^\infty$-functions $f:~\mathbb{R} \to \mathbb{C}$ which are compactly supported, i.e., which {\em{vanish identically outside some compact subset}} of $\mathbb{R}$. We write $f \in \mathcal{C}_c^\infty(\mathbb{R}; \mathbb{C})$ for such functions and define {\em{the support}} of $f$ by
% \begin{equation}
% \mathrm{supp}(f):= \overline{\{ x \in \mathbb{R} ~:~ f(x) \neq 0 \}} ~\mbox{;}
% \end{equation}
% the support of $f$ is thus the smallest {\underline{closed}} subset outside of which the function $f$ vanishes. 

% It's not obvious that compactly supported $\mathcal{C}^\infty$-functions exist at all: the non-trivial aspect here is that it is possible to force the function {\em{and all}} its derivatives to zero outside a fixed compact set. The subject of this problem is to show constructively that $\mathcal{C}_c^\infty(\mathbb{R}; \mathbb{C})$ is indeed non-empty and that there are in fact plenty such functions. These functions are collectively called ``$\mathcal{C}^\infty$-bump functions.'' As we will see during our upcoming discussion of the Fourier transform, $\mathcal{C}^\infty$-bump functions are extremely useful to ``smoothen out the edges of functions;'' for this reasons $\mathcal{C}^\infty$-bump functions are also known as {\em{mollifiers.}}




\vspace{0.1 cm}
\begin{itemize}
\item[(a)] %The basic ingredient to construct a $\mathcal{C}^\infty$-function of compact support is the following auxiliary function

\begin{prop}
    
    The function: 
    

\begin{equation}
h: \mathbb{R} \to \mathbb{R} ~\mbox{, } h(x) = \begin{cases} \mathrm{e}^{-1/x^2} & ~\mbox{, if $x > 0$ ,} \\ 0 & ~\mbox{, if $x \leq 0$ .} \end{cases}
\end{equation}

satisfies $h\in\mathcal{C}^\infty$.

\end{prop}

\begin{proof}
    
$$h:\mathbb{R}\rightarrow \mathbb{R},~h(x)=\begin{cases}e^{- \frac{1}{x^{2}}} & \text{if }x\ne0 \\
    0 & \text{if }x=0\end{cases}$$
    Show that $h$ is $\mathcal{C}^{\infty}$. 
    
    Let $P(n)$ be that $$h^{(n)}(x)=p_{n}\left(\frac{1}{x}\right)\cdot e^{-\frac{1}{x^{2}}},~\text{for all }x>0$$with $h^{(n)}(0)=0$ for some polynomial $p_{n}$.
    
    Base case: $n=0$. 
    Let $p_{0}=1$. We get $$h^{(0)}(x)=1\cdot h(x)=h(x),~\text{for all }x>0$$
    For $x=0$, the definition of $h$ gives $$h^{(0)}(0)=h(0)=0$$
    
    Inductive Step: Suppose $P(n)$ holds for some $n\ge0$ for a polynomial $p_{n}=\sum_{k=0}^{n_{k}}a_{k}x^{k}$. 
    
    For $x>0$, \begin{align*}
    h^{(n+1)}(x)&= \frac{d}{dx}h^{(n)}(x)\\
    &= \frac{d}{dx} \left(p_{n}\left(\frac{1}{x}\right)e^{- \frac{1}{x^{2}}}\right)\\
    &= \frac{d}{dx} \left(\sum_{k=0}^{n_{k}}a_{k}x^{-k}~e^{-\frac{1}{x^{2}}}\right)\\
    &= \sum_{k=0}^{n_{k}} a_{k}\frac{d}{dx}\left( x^{-k}~e^{-\frac{1}{x^{2}}}\right)\\
    &= \sum_{k=0}^{n_{k}} a_{k}\left( -kx^{-k-1}~e^{-\frac{1}{x^{2}}}+x^{-k}e^{- \frac{1}{x^{2}}}\cdot \frac{d}{dx} \left(\frac{-1}{x^{2}}\right)\right)\\
    &= \sum_{k=0}^{n_{k}} a_{k}\left( -kx^{-k-1}~e^{-\frac{1}{x^{2}}}+x^{-k}e^{- \frac{1}{x^{2}}}\cdot \frac{2}{x^{3}}\right)\\
    &= \sum_{k=0}^{n_{k}} a_{k}\left(-kx^{-k-1}~e^{-\frac{1}{x^{2}}}+x^{-k}e^{- \frac{1}{x^{2}}}\cdot \frac{2}{x^{3}}\right)\\
    &= \sum_{k=0}^{n_{k}} a_{k}\left(-k \left(\frac{1}{x}\right)^{k+1}~e^{-\frac{1}{x^{2}}}+2\left(\frac{1}{x}\right)^{k+3}e^{- \frac{1}{x^{2}}}\right)\\
    &= e^{- \frac{1}{x^{2}}}\sum_{k=0}^{n_{k}} a_{k}\left(-k \left(\frac{1}{x}\right)^{k+1}+2\left(\frac{1}{x}\right)^{k+3}\right)\\
    &:= e^{- \frac{1}{x^{2}}}p_{n+1}\left(\frac{1}{x}\right)
    \end{align*}
    
    
    At $x=0$, 
    
    $$h^{(n+1)}(0)=\lim_{x \rightarrow 0} \frac{h^{(n)}(x)-h^{(n)}(0)}{x-0}=\lim_{x \rightarrow 0} \frac{h^{(n)}(x)}{x}$$
    Since \begin{align*}
    &\lim_{x \rightarrow 0}h^{(n)}(x)\rightarrow 0\\
    &\lim_{x \rightarrow 0}x\rightarrow 0
    \end{align*}by the inductive hypothesis, the rule de l'Hopital says that this is indeed the limit: $$h^{(n+1)}(0)=\lim_{x \rightarrow 0} \frac{\frac{d}{dx}h^{(n)}(x)}{\frac{d}{dx}x}=\lim_{x \rightarrow 0} \frac{h^{(n+1)}(x)}{1}=\lim_{x\rightarrow 0}h^{(n+1)}(x)$$Note that $h^{(n)}$ differentiable for all $x>0$ was shown above, and this is what we need to compute the limit. 
    \vspace*{10  pt}
    
    On set 7, problem 4, we showed that $$x^{m}e^{- x^{2}}\rightarrow 0\text{ as }x \rightarrow \infty,\text{ for all }m\in \mathbb{N}_{0}$$
    Consider an arbitrary term $b_{k}x^{k}$ of $p_{n+1}$. Let $\epsilon>0$ be given and pick $x_{0}$ such that for all $x>x_{0}$, $$|x^{k}e^{- x^{2}}|< \frac{\epsilon}{b_{k}}$$
    If $0<x< \frac{1}{x_{0}}$, then, $$\frac{1}{x}>x_{0}$$and $$\left|b_{k} \left(\frac{1}{x}\right)^{k}e^{- \frac{1}{x^{2}}}\right|<\epsilon$$Thus, we see that each term of $p_{n+1}(x)e^{- \frac{1}{x^{2}}}\rightarrow 0$ as $x \rightarrow 0$ and thus $$h^{(n+1)}(x)= p_{n+1}(x)e^{- \frac{1}{x^{2}}}\rightarrow 0,\text{ as }x \rightarrow 0$$
    This concludes the inductive step.
    
    \vspace*{10 pt}
    By (PMI), we have $P(n)$ for all $n\in \mathbb{N}_{0}$. 
    
    
\end{proof}

% Show that $h$ is $\mathcal{C}^\infty$. Notice the ``miraculous property'' of $h$ is that it provides an example where you force a function {\em{and all its derivatives}} to vanish at a point (here at $x = 0$), but without the function itself being the zero function! 

% {\underline{Hint:}} From the algebra of derivatives, it is immediate that $h$ is infinitely often differentiable for $x \neq 0$ (see also (\ref{eq_represent}) below). The non-trivial aspect of the claim here is to prove that $h$ is also infinitely often differentiable at $x = 0$ and that
% \begin{equation} \label{eq_cinftybump}
% h^{(n)}(0) = 0  ~\mbox{, for all $n \in \mathbb{N}$.}
% \end{equation}
% You will have to use the definition of the derivative as a limit of the difference quotient to verify (\ref{eq_cinftybump}), i.e.
% \begin{equation} \label{eq_cinftybump_1}
% h^{(n)}(0) = \lim_{x \to 0} \dfrac{h^{(n-1)}(x) - h^{(n-1)}(0)}{x - 0} ~\mbox{, for } n \in \mathbb{N} ~\mbox{.}
% \end{equation} 
% To do so, first prove inductively that for all $n \in \mathbb{N}$, one can represent
% \begin{equation} \label{eq_represent}
% h^{(n)}(x) = p_n(1/x) \cdot \mathrm{e}^{-1/x^2} ~\mbox{,  for } x > 0 ~\mbox{,}
% \end{equation}
% for some polynomial $p_n$. The rule of de l'H\^opital will be useful to compute the limit in (\ref{eq_cinftybump_1}) using the representation of the derivative away from zero given in (\ref{eq_represent}).

\vspace{0.1 cm}
\item[(b)] 

\begin{prop}
The function
\begin{equation} \label{eq_bump}
g: \mathbb{R} \to \mathbb{R} ~\mbox{, } g(x):= \dfrac{h(x)}{h(x) + h(1 - x)} ~\mbox{.}
\end{equation}
satisfies $g \in \mathcal{C}^\infty(\mathbb{R}; \mathbb{R})$, $0 \leq g \leq 1$, and
\begin{equation}
g(x) = 0 ~\mbox{, if $x \leq 0$ , and } g(x) = 1 ~\mbox{, if $x \leq 1$.} 
\end{equation}
\end{prop}


\begin{proof}

    Recall the product rule:

\begin{align*}
(f\cdot g)^{(n)}=\sum_{k=0}^{n}\binom{n}{k}f^{(n-k)}g^{(k)}\\
\end{align*}


Let $f:\mathbb{R}\rightarrow \mathbb{R}$ be defined by $$f(x):=h(x)+h(1-x)$$so we have $g(x)= \frac{h(x)}{f(x)}$. 

\vspace*{10 pt}
Let $P(n)$ be that $g\in\mathcal{C}^{n}$. 

\vspace*{10 pt}

Base case: 

\vspace*{10 pt}

Note that $h(x)$ is 0 if and only if $x=0$. Also, $h$, is non-negative. Thus, $h(x)=-h(1-x)$ is false for all $x$ and $f$ is nonzero.

\vspace*{10 pt}

Since $f$ is nonzero and continuous, the quotient $h$ is continuous, by the algebra of continuous functions. This shows $P(0)$.

Inductive Step: Suppose $P(n)$ for some $n-1\ge0$. 

We have $g^{(k)}$ is continuous for all $k\le n-1$. Note that $f^{(k)}$ and $h^{(k)}$ are also continuous for all $k$ by part (a). 

\vspace*{10 pt}

Write $h= g\cdot f$ and apply the product rule:
\begin{align*}
h^{(n)}&= (f\cdot g)^{(n)}\\
&= \sum_{k=0}^{n}\binom{n}{k}f^{(n-k)}g^{(k)}\\
&= f\cdot g^{(n)}+\sum_{k=0}^{n-1}\binom{n}{k}f^{(n-k)}g^{(k)}\\
\iff g^{(n)}&= \frac{h^{(n)}-\sum_{k=0}^{n-1}\binom{n}{k}f^{(n-k)}g^{(k)}}{f}
\end{align*}
where we can divide by $f$ in the last step since it is nonzero. Thus, the algebra of continuous functions implies that $g^{(n)}$ is continuous. So, $g\in\mathcal{C}^{n}$.

\vspace*{10 pt}

By (PMI), we have $g\in\mathcal{C}^{\infty}$.
\end{proof}

\vspace{0.1 cm}
\item[(c)] %Use the function $g$ from part (b) to construct the following {\bf{``smooth-approximations'' of an indicator function}} for the interval $[-1,1]$: for each $\delta > 0$, construct a $\mathcal{C}^\infty$-function $\chi_\delta: \mathbb{R} \to \mathbb{R}$ which satisfies
% \begin{align}
% \chi_{\delta}: & ~\mathbb{R} \to \mathbb{C} ~\mbox{, } 0 \leq \chi_{\delta} \leq 1 ~\mbox{, } \nonumber \\
%  &\begin{cases} \chi_{\delta}(t) = 1 & ~\mbox{, for all $t \in [-1,1]$ ,} \\ \chi_{\delta}(t) = 0 & ~\mbox{, for all $t \in \mathbb{R} \setminus (-1-\delta, 1 + \delta)$ , } \\
%  0 < \chi_{\delta}(t) < 1 & ~\mbox{, for all $1 < \vert t \vert < 1+ \delta$ ,}  \end{cases}
% \end{align}
% in particular, 
% \begin{equation}
% \mathrm{supp}(\chi_{\delta}) = [-1-\delta, 1+ \delta] ~\mbox{.}
% \end{equation}

% Use \url{https://www.desmos.com/calculator} to sketch the graph of these functions for three values of $\delta$. Include these graphs as a figure to this problem.

% \vspace{0.1 cm}
% {\underline{NOTE:}} To include the sketch of your graphs in your LaTeX document, the TeX file for the prompt to this set includes a command with instructions of how to do that.

Let $\chi_\delta:\mathbb{R}\to\mathbb{C}$ be defined by:

\begin{equation}
    \chi_{\delta}(x):=\begin{cases} 
        0 & \text{if }t\notin(-1-\delta,1+\delta) \\
    1 & \text{if }t\in[-1,1] \\
    g\left(\frac{1+\delta-x}{\delta}\right) & \text{if }t\in(1,1+\delta) \\
    g\left(\frac{x+1+\delta}{\delta}\right) & \text{if }t\in(-1-\delta,-1)
    \end{cases}
\end{equation}

For all $t\in[-1,1]$, $$\chi_{\delta}(t)=1$$
For all $t\notin(-1-\delta,1-\delta)$, $$\chi_{\delta}(t)=0$$
If $1<t<1+\delta$, $$0<\frac{1+\delta-t}{\delta}<1$$
So, $$0<g(t)<1$$
If $-1-\delta<t<-1$, $$0< \frac{x+1+\delta}{\delta}<1$$So, $$0<g(t)<1$$


\vspace*{10 pt}

Clearly, $\chi_{\delta}$ is $\mathcal{C}^{\infty}$ except possibly at $t=-1-\delta,-1,1,1+\delta$. Noting that $\chi_{\delta}$ is even, we consider $t=1,1+\delta$. 

At $t=1$, for all $n>0$, the left limit of $\chi_{\delta}^{(n)}$ is $0$, since $\chi_{\delta}$ is constant left of 1. The right limit is also 0, since we have $$\frac{1+\delta-1}{\delta}=1$$and, b implies that $g^{(n)}$ is 0 at 1. For $n=0$, $\chi_{\delta}$ is uniformly 1 to the right of 1 (and at 1). Also, $g(1)=1$.

At $t=1+\delta$, for all $n>0$, the left limit of $\chi_{\delta}^{(n)}$ is 0, since we have $$\frac{1+\delta-1-\delta}{\delta}=0$$and, $b$ implies that $g^{(n)}$ is 0 at $0$. The right limit is 0, since $\chi_{\delta}$ is constant right of $1+\delta$. For $n=0$, $\chi_{\delta}$ is uniformly 0 to the right of $1+\delta$. Also, $g(0)=0$.

Here is a graph of $\chi_\delta$ for $\delta=.4$, $\delta=.9$, and $\delta=2$: 

\begin{figure}[h]
    
    \includegraphics*[width = 6 in]{Screenshot 2024-04-15 at 8.34.09 PM.png}

\end{figure}




%\begin{center}        
%       \includegraphics[width= 0.8 \columnwidth]{fig.pdf}
%\end{center} 

%%Follow these steps: 
% 1. Produce a pdf of the desmos graphs: Desmos allows you to pdf-print any graph. If needed, crop the file to a size that is not too big. Save your pdf file as fig.pdf
% 2. Upload your pdf to overleaf, in the same folder as the TeX file for your solutions
% 3. You can change the size of your figure by changing the number in the width command; currently this is set to 0.8 \columnwidth, i.e. the 0.8 can be changed to any number \in [0,1].
% 4. Copy the command above into your TeX file and remove the % at the beginning of the three lines of the command. Any line starting with % is treated by LaTeX as a comment, i.e. it is not taken into account when compiling the file.
\end{itemize}

\vspace{0.2 cm}

\item {\bf{Problem 4:}} This problem follows up on our discussion of the {\bf{Steinhaus three distance (aka as three gap) theorem}} from class and its proof, which was the part of the reading assignment for Monday, 04/15. 

\vspace{0.1 cm}
\begin{itemize}
% \item[(a)] Read again Peter Shiu's short proof of the Steinhaus three distance theorem which appeared in the {\em{Monthly}} of the American Mathematical Society in 2018. The paper is posted in the folder \texttt{Course Material/week 9} on the \texttt{blackboard} page of the class. {\em{There is nothing to turn in here.}}

% \vspace{0.1 cm}
% \underline{COMMENT:} You will notice that the style of writing in a research paper is much more terse than what you are used to from undergraduate textbooks in mathematics. To keep papers concise and focussed on the important ideas, small details and/or computations are often left to reader. Hence, reading a research paper requires an active reader, ready with paper and pencil to fill in these details as they read the article. 

\vspace{0.1 cm}
\item[(b)] %In view of the Shiu's proof of the Steinhaus three distance theorem from the reading in part (a), {\bf{supply the details for the following claim}} from the paper, which plays a key role in the proof of Lemma 3. The statement of the claim here uses the notation and language from the paper:

\begin{prop}
For irrational $\alpha$, it is impossible that every gap has length either $g_m$ or $2 g_m$.
\end{prop}


\begin{proof}
    Suppose that every gap has length either $g_{m}$ or $2g_{m}$. We first show that $g_{m}$ must be rational.
\begin{align*}
\sum_{k=1}^{n}g_{k}&= 1+b_{1}-b_{n}+\underbrace{\sum_{k=2}^{n}b_{k}-b_{k-1}}_{\text{telescopes}}\\
&= 1+b_{1}-b_{n} -b_{1}+b_{n}\\
&= 1
\end{align*}
Note also that each $g_{k}$ is an integer multiple of $g_{m}$. In particular, there exists $k\in \mathbb{N}$ such that $$1=\sum_{k=1}^{n}g_{k}=kg_{m}$$So, $$g_{m}= \frac{1}{k}\in\mathbb{Q}$$

Consider two cases: 
Case 1: $b_{n}$ is the first iterate corresponding to the fractional part of $1\alpha$. Then, $b_{n}=\alpha$. Since $n>1$, we have some $k<n$ such that $$b_{n}+\alpha= b_{k}(\mod1)$$Note that since $b_{k}<b_{n}$, $$b_{n}+\alpha-1=b_{k}\iff b_{n}-b_{k}=1-\alpha$$
For the points in between $b_{k}$ and $b_{n}$, we count the number of each distance: 
\begin{align*}
l:=&\#\{i:k\le i<n,~b_{i+1}-b_{i}=g_{m}\}\\
j:=&\#\{i:k\le i<n,~b_{i+1}-b_{i}=2g_{m}\}
\end{align*}
As in part b, a telescoping sum argument shows that the total distance between $b_{k}$ and $b_{n}$ is given by $$1-\alpha=b_{n}-b_{k}=lg_{m}+2jg_{m}$$
Thus, $$1-\alpha=(\underbrace{l+2j}_{\in\mathbb{Q}})g_{m}$$and so $\alpha\in \mathbb{Q}$ by the closure of $\mathbb{Q}$ under multiplication and addition.


Case $2$: $b_{n}$ is not the first iterate. Then, $b_{n}>\alpha$, so there exists $k$ with $b_{k}=b_{n}-\alpha$ and $n>k$. We use the same argument as in Case 1, calculating the total distance between $b_{k}$ and $b_{n}$ as an integer multiple of $g_m$ to show that $\alpha\in \mathbb{Q}$.
\end{proof}

% \vspace{0.1 cm}
% This claim has relevance for musical scales: Based on above claim, for every $7$-tone scale generated by a fixed irrational, one {\bf{cannot have}} the situation:
% \begin{equation}
% \mbox{wholetone} = 2 \times \mbox{semitone} ~\mbox{.}
% \end{equation}

\vspace{0.1 cm}
\item[(c)] %In class we formulated the Steinhaus three gap theorem from the point of view of musical scales in the following language:
\begin{theorem}[Steinhaus three gap theorem - musical scale version] \label{music_TGT}
Let $\alpha > 0$ be a fixed irrational generator. Consider a musical scale of length $q \in \mathbb{N}$ generated by $\alpha$, given by the set of $q$ distinct points (=pitches) of the form
\begin{equation} \label{eq_scale}
\mathcal{S}_q(k):=\{ R_\alpha^j(0) ~\mbox{:} ~ k \leq j \leq k + q -1 \} ~\mbox{, where $k \in \mathbb{Z}$ .}
\end{equation}

Then, one has:
\begin{itemize}
\item[(i)] The set $\mathcal{S}_n(k)$ will give rise to two or three distinct, consecutive (pitch) distances.
\item[(ii)] Scales with two consecutive (pitch) distances arise if and only if $q$ corresponds to an optimal scale size of the first kind. Here, recall from class that optimal scales of the first kind are precisely those for which the ``{\em{discarded}}'' $(q+1)$st pitch in the tuning process giving rise to the scale in (\ref{eq_scale}), i.e., the iterate
\begin{equation}
R_\alpha^j(0) ~\mbox{, for $j = k + q$ ,}
\end{equation}
is a neighboring iterate for the scale's origin at $R_\alpha^k(0)$.
\end{itemize}
\end{theorem}


\begin{proof}
    Let's generalize the definition of a scale: 
$$\mathcal{S}_{n}(k,x):=\{R_{\alpha}^{j}(x):k\le j\le n+k-1\}$$
Note that we wish to show that (i) and (ii) hold for all scales of type: 
\begin{align*}
\mathcal{S}_{n}(k,0)&=\{R_{\alpha}^{j}(0):k\le j\le n+k-1\}\\
&= \{\underbrace{R^{i+k-1}_{\alpha}(0)}_{=R^{i}_{\alpha}(R^{k-1}_{\alpha}(0))}:1\le i\le n\}\\
&= \mathcal{S}_{n}(1,R_{\alpha}^{k-1}(0))
\end{align*}
We will show that (i) and (ii) hold for all scales of the form $$\mathcal{S}_{n}(1,x):x\in[0,1)$$which implies Theorem \ref{music_TGT}. 
\vspace*{10 pt}

Consider the sequence $b_{m}$ of Shiu's formula with $$0<b_{1}<\cdots b_{n}<1$$and let $x\in [0,1)$ be given. Let $m$ be 1 if $b_{1}+x\ge1$. Otherwise, we take $m$ to be the greatest integer such that $b_{m}+x<1$:
$$
m:=\max\{k\in \mathbb{N}:b_{k}+x<1\}
$$
Observe that the sequence 
$$\{a_{k}\}_{k=1}^{n}:=\{b_{k}+x(\mod 1)\}_{k=1}^{n}$$
may be written in increasing order: 
$$
0<a_{m+1}<\cdots<a_{n}<a_{1}<\cdots<a_{m}<1
$$
since: 
\begin{align*}
k\le m&\implies b_{k}\le b_{m}\\
&\implies b_{k+x}\le b_{m}+x<1\\
&\implies a_{k}=b_{k}+x\\
k>m&\implies b_{m}<b+k \\
&\implies b_{k}+k \not<1\\
&\implies a_{k}=b_{k}+x-1<a_{1}
\end{align*}
So we see that the order of $a_{1}\ldots, a_{m}$ is the same and the order of $a_{m+1},\ldots a_{n}$ is the same, but with the larger terms first.

\vspace*{10 pt}

We compute the gaps of $a_{k}$: $$G_{1}=1+a_{1}-a_{n},\quad G_{m}=a_{m}-a_{m-1},\quad m=2,\ldots,n$$
\begin{align*}
G_{1}&= 1+a_{1}-a_{n}\\
&= 1+[(b_{1}+x)\mod1]-[(b_{n}+x)\mod1]\\
&= 1+[(b_{1}+x)-(b_{n}+x)\mod1]\\
&= 1+[(\underbrace{b_{1}-b_{n}}_{<1})\mod1]\\
&= 1+b_{1}-b_{n}\\
&= g_{1}\\
G_{m}&=a_{m}-a_{m-1}\\
&= [(b_{m}+x)\mod1]-[(b_{m-1}+x)\mod1]\\
&= [(b_{m}+x)-(b_{m-1}+x)]\mod1\\
&= (\underbrace{b_{m}-b_{m-1}}_{<1})\mod1\\
&= b_{m}-b_{m-1}\\
&= g_{m}
\end{align*}
Thus, the sequence $a_{k}$ has exactly the same gaps as $b_{m}$. 

Note that $$\{a_{k}\}_{k=1}^{n}=\mathcal{S}_{n}(1,x)$$
since for each $1\le k\le n$, we can write $$a_{k}=(j \alpha\mod1)+x\mod1=j \alpha+x\mod 1= R^{j}_{\alpha}(x)$$for a unique $j$ (uniqueness from Kronecker's Theorem). This concludes the argument of (i) for $\mathcal{S}_{n}(k)$.

\vspace*{10 pt}

We see that the gaps of the scale $\mathcal{S}_{n}(1,x)$ are independent of $x$. In particular, the gaps of $\mathcal{S}_{n}(k)$ are the same for all $k\in \mathbb{N}$. We consider the case of $\mathcal{S}_{n}(1)$. Notice that the elements of $\mathcal{S}_{n}(1)$ form exactly the sequence $\{b_{m}\}_{m=1}^{n}$. Also it is best scale of the first kind if and only if either the first or $n$-th iterate is closest to 0. This is equivalent to Shiu's definition of $n$ being a node. So, the lemma gives (ii) for $\mathcal{S}_{n}(1)$ in particular. By the earlier comment, we see that it holds generally for $\mathcal{S}_{n}(k)$.     
\end{proof}

% Explain how the statement of the Steinhaus three gap theorem in Peter Shiu's article implies the musical formulation given above. Notice that, translating the musical language above, Peter Shiu's article considers the special case of the set $\mathcal{S}_n(1)$. {\em{Your argument should not reprove the theorem from scratch, but rather explain how the proof by Shiu implies the statement above. {\bf{This should be relatively quick; use the hint below.}}}}

% \vspace{0.1 cm}
% {\underline{Hint:}} For $k \neq 1$, how can you generate $\mathcal{S}_n(k)$ from $\mathcal{S}_n(1)$? What does this imply for the relation between the set of distances?

\end{itemize}

\end{itemize}

\end{document}

%------------------------------------------------------------------------------
% End of journal.tex
%------------------------------------------------------------------------------
